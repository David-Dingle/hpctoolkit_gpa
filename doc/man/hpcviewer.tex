%% $Id$

%%%%%%%%%%%%%%%%%%%%%%%%%%%%%%%%%%%%%%%%%%%%%%%%%%%%%%%%%%%%%%%%%%%%%%%%%%%%%
%%%%%%%%%%%%%%%%%%%%%%%%%%%%%%%%%%%%%%%%%%%%%%%%%%%%%%%%%%%%%%%%%%%%%%%%%%%%%

\documentclass[english]{article}
\usepackage[latin1]{inputenc}
\usepackage{babel}
\usepackage{verbatim}

%% do we have the `hyperref package?
\IfFileExists{hyperref.sty}{
   \usepackage[bookmarksopen,bookmarksnumbered]{hyperref}
}{}

%% do we have the `fancyhdr' or `fancyheadings' package?
\IfFileExists{fancyhdr.sty}{
\usepackage[fancyhdr]{latex2man}
}{
\IfFileExists{fancyheadings.sty}{
\usepackage[fancy]{latex2man}
}{
\usepackage[nofancy]{latex2man}
\message{no fancyhdr or fancyheadings package present, discard it}
}}

%% do we have the `rcsinfo' package?
\IfFileExists{rcsinfo.sty}{
\usepackage[nofancy]{rcsinfo}
\rcsInfo $Id$
\setDate{\rcsInfoLongDate}
}{
\setDate{2022/01/10}
\message{package rcsinfo not present, discard it}
}

\setVersionWord{Version:}  %%% that's the default, no need to set it.
\setVersion{=PACKAGE_VERSION=}

%%%%%%%%%%%%%%%%%%%%%%%%%%%%%%%%%%%%%%%%%%%%%%%%%%%%%%%%%%%%%%%%%%%%%%%%%%%%%
%%%%%%%%%%%%%%%%%%%%%%%%%%%%%%%%%%%%%%%%%%%%%%%%%%%%%%%%%%%%%%%%%%%%%%%%%%%%%

\begin{document}

\begin{Name}{1}{hpcviewer}{The HPCToolkit Performance Tools}{The HPCToolkit Performance Tools}{hpcviewer:\\ Interactive Presentation of Performance}

The Java-based \Prog{hpcviewer} interactively presents program performance in a top-down fashion.

\end{Name}

%%%%%%%%%%%%%%%%%%%%%%%%%%%%%%%%%%%%%%%%%%%%%%%%%%%%%%%%%%%%%%%%%%
\section{Synopsis}

Command-line usage:\\
\SP\SP\SP\Prog{hpcviewer} \oOpt{options} \oOpt{hpctoolkit-database}

GUI usage:\\
\SP\SP\SP Launch \File{hpcviewer} and open the experiment database \oOpt{hpctoolkit-database}.


%%%%%%%%%%%%%%%%%%%%%%%%%%%%%%%%%%%%%%%%%%%%%%%%%%%%%%%%%%%%%%%%%%
\section{Description}

The Java-based \Prog{hpcviewer} interactively presents program-performance experiment databases
in a top-down fashion.
Since experiment databases are self-contained,
they may be relocated from a cluster for visualization on a laptop or workstation.

%%%%%%%%%%%%%%%%%%%%%%%%%%%%%%%%%%%%%%%%%%%%%%%%%%%%%%%%%%%%%%%%%%
\section{Arguments}

\begin{Description}
\item[\Arg{hpctoolkit-database}] An HPCToolkit experiment database
produced by \Prog{hpcprof}.
\end{Description}


\subsection{Options}

\begin{Description}

\item[\Opt{-h} \Opt{--help}]
Print a help message.


\item[\Opt{-jh}, \Opt{--java-heap} <size>]
  	Set the JVM maximum heap size for this execution of  \Prog{hpcviewer}. The value of \texttt{size} must be
	in megabytes (M) or gigabytes (G). For example, one can specify a \texttt{size}  of 3 gigabytes as either
	3076M or 3G.


\end{Description}


%%%%%%%%%%%%%%%%%%%%%%%%%%%%%%%%%%%%%%%%%%%%%%%%%%%%%%%%%%%%%%%%%%
\section{Profile view}

\subsection{Views}

\Prog{hpcviewer} supports four principal views of an application's performance data.
Each view reports both inclusive costs (including callees) and exclusive costs (excluding callees).

\begin{itemize}

\item \textbf{Top-down view.}
A top-down view depicting the dynamic calling contexts (call paths) in which costs were incurred.
With this view one can explore the performance of an application in a top-down manner
to understand the costs incurred by calls to a procedure in a particular calling context.

\item \textbf{Bottom-up view.}
This bottom-up view enables one to look upward along call paths.
It apportions a procedure's costs to its caller and, more generally,
its calling contexts.
This view is particularly useful for understanding the performance of software components or procedures
that are used in more than one context.

\item \textbf{Flat view.}
This view organizes performance data according to the static structure of an application.
All costs incurred in \emph{any} calling context by a procedure are aggregated together in the flat view.

\item \textbf{Thread view}
	This view is to display the metrics of a certain threads and/or processes.
To activate the view, one needs to select a thread or a set of threads of interest by clicking the thread-view button from Top-down view.
On the thread selection window, one needs to select the checkbox of the threads of interest.
To  narrow the list, one can specify the thread name on the filter part of the window.
Once threads have been selected, click \textbf{OK}, and the Thread view will be activated.
The tree of the view is the same as the tree from the Top-down view, with the metrics only from the selected threads.
If there are more than one selected threads, the metrics are the average of the values of the selected threads.
\\
Currently contexts for GPU streams are not shown in the thread view; metrics for a GPU operation are associated with the calling context in the thread that initiated the GPU operation.


\end{itemize}


\subsection{Panes}

The browser window is split into two panes:

\begin{itemize}

\item \textbf{Source pane.} The source pane appears at the top of the browser window
and displays source code associated with the current selection in the navigation pane below.
Making a selection in the navigation pane causes the source pane
to display the selection's corresponding source file and highlight the source line.

\item \textbf{Navigation pane.}
The navigation pane appears at the bottom left of the browser window
and displays an outline (tree structure) organizing the performance measurements under investigation.
Each item in the outline denote a structure in the source code such as a
load module, source file, procedure, procedure activation,
loop, single line of code, or code fragment inlined from elsewhere.
Outline items can be selected and their children folded and unfolded.

Which items appear in the outline depend on which view is displayed:

\begin{itemize}

\item In the Top-down view, displayed items are
procedure activations, loops, source lines, and inlined code.
Most items link to a single location in the source code,
but a procedure activation item links to two:
the call site where the procedure was invoked and the procedure body executed in response.

\item In the Bottom-up view, displayed items are always procedure activations.
Unlike the Top-down view, where a call site is paired with its called procedure,
in this view a call site is paired with its calling procedure,
attributing costs for a called procedure among all its call sites (and therefore callers).

\item In the flat view, displayed items are
source files, call sites, loops, and source lines.
Call sites are rendered in the same way as procedure activations.

\end{itemize}

The header above the navigation pane contains buttons for adjusting the displayed view:

\begin{itemize}

\item \textbf{Up arrow.} \emph{Zoom in} to show only information for the selected line and its descendants.

\item \textbf{Down arrow.} \emph{Zoom out} to reverse a previous zoom-in operation.

\item \textbf{Hot path}. Toggle hot path mode,
which automatically unfolds subitems along the \emph{hot path} for the currently selected metric:
those subitems encountered by starting at the selected item
and repeatedly descending to the child item with largest cost for the metric.
This is an easy way to find performance bottlenecks for that metric.

\item \textbf{Derived metric}. Define a new metric in terms of existing metrics
by entering a spreadsheet-style formula.

\item \textbf{Filter metrics}. Show the metric property view which allows to show or hide specified metrics of the current table.
One can also edit the name of a metric column or even edit the formula of a derived metric.

\item \textbf{CSV export}. Write data from the current table to a file
in standard CSV (Comma Separated Values) format.

\item \textbf{Bigger text}. Increase the size of displayed text.

\item \textbf{Smaller text}. Decrease the size of displayed text.

\item \textbf{Showing graph of metric values}.
Showing the graph (plot, sorted plot or histogram) of metric values of the selected node in CCT for all processes or threads.

\item \textbf{Show the metrics of a set of threads}.
Showing the CCT and the metrics of a seletected threads.



\item \textbf{Flatten} (icon of a slashed tree node).
\emph{Flatten} the navigation pane outline,
i.e. replace each top-level item by its child subitems
(available in flat view only).
If an item has no children it remains in the outline.
Flattening may be performed repeatedly, each step hiding another level of the outline.
This is useful for relaxing the strict hierarchical view
so that peers at the same level in the tree can be viewed and ranked together.
For instance, this can be used to hide procedures in the flat view
so that outermost loops can be ranked and compared.

\item \textbf{Unflatten.} Undo one previous flatten operation (flat view only).

\end{itemize}

\item  \textbf{Metric pane.}
The metric pane appears to the right of the navigation pane at the bottom of the window
and displays one or more columns of performance data, one metric per column.
Each row displays measured metric values for the source structure denoted by the outline item to its left.
A metric may be selected by clicking on its column header,
causing outline items at each level of the hierarchy to be sorted by their values for that metric.

\end{itemize}


\subsection{Thread-Centric Graphs}

\Prog{hpcviewer} can display graphs of thread-level metric values.
This is useful for quickly assessing load imbalance across processes and threads.

To create a graph,
choose the calling context view and select an item in the navigation pane,
then pop up the context menu by right-clicking the item.
A list of graphable metrics appears at the bottom of the context menu,
each with a sub-menu showing the three graph styles that \Prog{hpcviewer} can make.
The \emph{Plot} graph displays metrics by MPI rank and thread number;
The \emph{Sorted plot} graph displays metrics sorted by value;
and the \emph{Histogram} graph displays a barchart of metric value distributions.




% ===========================================================================
% ===========================================================================


\section{Trace view}

The view interactively presents program traces in a top-down fashion.
It comprises of three different parts.


\begin{itemize}
\item \textbf{Main view} (left, top):
  This is the primary view.
  This view, which is similar to a conventional process/time (or space/time) view, shows time on the horizontal axis and process (or thread) rank on the vertical axis; time moves from left to right.
  Compared to typical process/time views, there is one key difference.
  To show call path hierarchy, the view is actually a user-controllable slice of the process/time/call-path space.
  Given a call path depth, the view shows the color of the currently active procedure at a given time and process rank.
  (If the requested depth is deeper than a particular call path, then the viewer simply displays the deepest procedure frame and, space permitting, overlays an annotation indicating the fact that this frame represents a shallower depth.)
  hpcviewer assigns colors to procedures based on (static) source code procedures.
  Although the color assignment is currently random, it is consistent across the different views.
  Thus, the same color within the Trace and Depth Views refers to the same procedure.
  The Trace View has a white crosshair that represents a selected point in time and process space.
  For this selected point, the Call Path View shows the corresponding call path.
  The Depth View shows the selected process.

\item \textbf{Depth view} (left, bottom):
  This is a call-path/time view for the process rank selected by the Trace view's crosshair.
  Given a process rank, the view shows for each virtual time along the horizontal axis a stylized call path along the vertical axis, where `main' is at the top and leaves (samples) are at the bottom.
  In other words, this view shows for the whole time range, in qualitative fashion, what the Call Path View shows for a selected point.
  The horizontal time axis is exactly aligned with the Trace View's time axis; and the colors are consistent across both views.
  This view has its own crosshair that corresponds to the currently selected time and call path depth.

\item \textbf{Summary view} (same location as depth view on the left-bottom part):
  The view shows for the whole time range displayed, the proportion of each subroutine in a certain time.
  Similar to Depth view, the time range in Summary reflects to the time range in the Trace view.

\item \textbf{Call stack view} (right, top):
  This view shows two things: (1) the current call path depth that defines the hierarchical slice shown in the Trace View; and (2) the actual call path for the point selected by the Trace View's crosshair.
  (To easily coordinate the call path depth value with the call path, the Call Path View currently suppresses details such as loop structure and call sites; we may use indentation or other techniques to display this in the future.)

\item \textbf{Statistics view} (tab in top, right pane)
  The view shows a list of procedures and the estimated execution percentage for each for the time interval currently shown in the Trace view.


\item \textbf{Mini map view} (right, bottom):
  The Mini Map shows, relative to the process/time dimensions, the portion of the execution shown by the Trace View.
  The Mini Map enables one to zoom and to move from one close-up to another quickly.

\end{itemize}
Note:
\begin{itemize}
\item GPUs are very fast, hence the time interval during which a GPU operation is active may be very short. A problem for users is that it may be hard to locate short GPU operations that are separated by long intervals of idleness in the trace. Such operations will often be invisible because
 when hpcviewer renders a pixel in a trace, it will not show a GPU operation unless the time point at the left edge of the pixel's associated time interval falls within the time interval of the GPU operation.
 	To force hpcviewer to render a GPU operation if any GPU operation is active within the time interval associated with a pixel, one can enable \emph{Expose GPU traces} by clicking the menu \textbf{File} - \textbf{Preferences} and click the \textbf{Traces} page, then check the \emph{Expose GPU traces} option.
\\
Warning: enabling this option causes trace statistics to be unreliable because GPU activity will be overrepresented.
\end{itemize}



% ===========================================================================

\subsection{Main view}

Main view is divided into two parts: the top part which contains \emph{action pane} and the \emph{information pane}, and the main view which displays the traces.

The buttons in the action pane are the following:
\begin{itemize}

\item Home : Resetting the view configuration into the original view, i.e., viewing traces for all times and processes.
\item Horiontal zoom in / out : Zooming in/out the time dimension of the traces.
\item Vertical zoom in / out : Zooming in/out the process dimension of the traces.
\item Navigation buttons : Navigating the trace view to the left, right, up and bottom, respectively. It is also possible to navigate with the arrow keys in the keyboard. Since Trace view does not support scrool bars, the only way to navigate is through navigation buttons (or arrow keys).
\item Undo : Canceling the action of zoom or navigation and returning back to the previous view configuration.
\item Redo : Redoing of previously undo change of view configuration.
\item Save/Load a view configuration : Saving/loading a saved view configuration.
A view configuration file contains the information of the current dimension of time and process, the depth and the position of the crosshair.
It is recommended to store the view configuration file in the same directory as the database to ensure that the view configuration file matches well with the database since the file does not store which database it is associated with.
Although it is possible to open a view configuration file which is associated from different database, it is highly not recommended since each database has different time/process dimensions and depth.


\end{itemize}

The information pane contains some information concerning the range status of the current displayed data.
\begin{itemize}
 \item Time Range. The information of current time-range (horizontal) dimension.
 \item Rank Range. The information of current process-range (vertical) dimension.
 \item Cross Hair. The information of current crosshair position in time and process dimensions.
\end{itemize}



% ===========================================================================
% ===========================================================================
\subsection{Depth view}

Depth view shows all the call path for a certain time range [t_1,t_2]= \{t | t_1 <= t <= t_2\} in a specified process rank $p$. The content of Depth view is always consistent with the position of the cross-hair in Trace view.
For instance once the user clicks in process $p$ and time $t$, while the current depth of call path is $d$, then the Depth view's content is updated to display all the call path of process $p$ and shows its cross-hair on the time $t$ and the call path depth $d$.

On the other hand, any user action such as cross-hair and time range selection in Depth view will update the content within Trace view. Similarly, the selection of new call path depth in Call view invokes a new position in Depth view.

In Depth view a user can specify a new cross-hair time and a new time range.

\textbf{Specifying a new cross-hair time.} Selecting a new cross-hair time $t$ can be performed by clicking a pixel within Depth view. This will update the cross-hair in Trace view and the call path in Call view.

\textbf{Selecting a new time range.} Selecting a new time range [t_m,t_n]= \{t | t_m <= t <= t_n\} is performed by first clicking the position of $t_m$ and drag the cursor to the position of $t_n$. A new content in Depth view and Trace view is then updated. Note that this action will not update the call path in Call view since it does not change the position of the cross-hair.


% ===========================================================================
% ===========================================================================
\subsection{Summary view}

Summary view presents the proportion of number of calls of time $t$ across the current displayed rank of process $p$.
Similar to Depth view, the time range in Summary view is always consistent with the time range in Trace view.
One can also select a new time range in this view.


% ===========================================================================
% ===========================================================================
\subsection{Call view}

This view lists the call path of process \textbf{p} and time \textbf{t} specified in Trace view and Depth view.
This view can show a call path from depth $0$ to the maximum depth, and the current depth is shown in the depth editor (located on the top part of the view).

In this view, the user can select the depth dimension by either typing the depth in the depth editor or selecting a procedure in the table of call path.


% ===========================================================================
% ===========================================================================
\subsection{Statistic view}

The view shows a list of procedures and the estimated execution percentage for each for the time interval currently shown in the Trace view.
  Whenever the user changes the time interval displayed in the Trace view, the statistics view will update its list of procedures and their execution percentages to
  reflect the current interval.  Similarly, a change in the selected call path depth will also update the contents of the statistics view.


% ===========================================================================
% ===========================================================================
\subsection{Mini view}

The Mini view shows, relative to the process/time dimensions, the portion of the execution shown by the Trace view.
In Mini view, the user can select a new process/time (p_a,t_a),(p_b,t_b) dimensions by clicking the first process/time position (p_a,t_a) and then drag the cursor to the second position (p_b,t_b).
The user can also moving the current selected region to another region by clicking the white rectangle and drag it to the new place.




% ===========================================================================
% ===========================================================================

\section{Menus}

\Prog{hpcviewer} provides five main menus:

% ==========================================================
% ==========================================================

\subsection{File}
This menu includes several menu items for controlling basic viewer operations.
\begin{itemize}
\item \textbf{New window}
  Open a new \Prog{hpcviewer} window that is independent from the existing one.

\item \textbf{Switch database}
  Load a performance database into the current \Prog{hpcviewer} window and close all opened databases (if any).

\item \textbf{Open database}
  Load a performance database into the current \Prog{hpcviewer} window.
  Currently \Prog{hpcviewer} restricts maximum of five database open at a time.
  To display more, one can either closing an existing open database, or opening a new \Prog{hpcviewer} window.

\item \textbf{Close database}
  Unloading a performance database.

\item \textbf{Merge database}
  Merging two database that are currently in the viewer. If \Prog{hpcviewer} has more than two
  open database, then one needs to choose which database to be merged.
  Currently \Prog{hpcviewer} does not support storing a merged database into a file.

  \begin{itemize}
   \item \textbf{Merge top-down tree} Merging the top-down trees of the two opened database.
   \item \textbf{Merge flat tree} Merging the flat trees of the two opened database.
  \end{itemize}

\item \textbf{Preferences...}
  Display the settings dialog box which consists of three sections:
  \begin{itemize}
     \item \textbf{Appearance} Change the fonts for tree and metric columns and source viewer.
     \item \textbf{Traces} Specify settings for Trace view such as the number of working threads to be used and the tooltip's delay.
     \item \textbf{Debug} Enable/disable debug mode.
  \end{itemize}

\item \textbf{Exit}
  Quit the \Prog{hpcviewer} application.

\end{itemize}

% ==========================================================
% ==========================================================

\subsection{Filter}
This menu is to allow users to filter certain nodes in the Profile view or filter certain profiles in the Trace view.
\begin{itemize}
 \item \textbf{Filter CCT nodes}.
  Open a filter property window which lists a set of filters and its properties.
\Prog{hpcviewer} allows  users to define multiple filters, and each filter is associated with a type and a glob pattern (A glob pattern specifies which name to be removed by using wildcard characters such as *, ? and +).
There are three types of filter: ``\textbf{self only}'' to omit matched nodes,
``\textbf{descendants only}'' to exclude only the subtree of the matched nodes, and ``\textbf{self and descendants}'' to
remove matched nodes and its descendants.

 \item \textbf{Filter ranks} \emph{(Trace view mode only)}.
  Open a window for selecting which ranks should be displayed.

\end{itemize}

% ==========================================================
% ==========================================================

\subsection{View}
This menu is only visible if at least one database is loaded.
All actions in this menu are intended primarily for tool developer use.
By default, the menu is hidden. Once a database is loaded, the menu is then visible.

\begin{itemize}
 \item \textbf{Show metric}
Show the metric property view which allows to show or hide specified metrics of the current table.
One can also edit the name of a metric column or even edit the formula of a derived metric.

 \item \textbf{Split window}
 Enabled if there are two databases open. This menu allows to split vertically two databases into two panes to easily compare them.

 \item \textbf{Color map} \emph{Trace view only}: to open a window which shows customized mapping between a procedure pattern and a color. \Prog{hpcviewer} allows users to customize assignment of a pattern of procedure names with a specific color.

 \item \textbf{Debug} \emph{(if the debug mode is enabled)}
 A special set of menus for advanced users when the debug mode is enabled. The menu is useful to debug \Prog{hpcviewer}. The menu consists of:

   \begin{itemize}
     \item \textbf{Show database raw's XML}
 	Enable one to request display of raw XML representation for performance data.
  \end{itemize}

\end{itemize}

% ==========================================================
% ==========================================================

\subsection{Help}

This menu displays information about the viewer.
\begin{itemize}

\item \textbf{About}.
  Displays brief information about the viewer, including JVM and Eclipse variables, and error log files.

\end{itemize}



%%%%%%%%%%%%%%%%%%%%%%%%%%%%%%%%%%%%%%%%%%%%%%%%%%%%%%%%%%%%%%%%%%
%\section{Examples}

%%%%%%%%%%%%%%%%%%%%%%%%%%%%%%%%%%%%%%%%%%%%%%%%%%%%%%%%%%%%%%%%%%
%\section{Notes}

%%%%%%%%%%%%%%%%%%%%%%%%%%%%%%%%%%%%%%%%%%%%%%%%%%%%%%%%%%%%%%%%%%
\section{See Also}

\HTMLhref{hpctoolkit.html}{\Cmd{hpctoolkit}{1}}.

%%%%%%%%%%%%%%%%%%%%%%%%%%%%%%%%%%%%%%%%%%%%%%%%%%%%%%%%%%%%%%%%%%
\section{Version}

Version: \Version

%%%%%%%%%%%%%%%%%%%%%%%%%%%%%%%%%%%%%%%%%%%%%%%%%%%%%%%%%%%%%%%%%%
\section{License and Copyright}

\begin{description}
\item[Copyright] \copyright\ 2002-2022, Rice University.
\item[License] See \File{LICENSE}.
\end{description}

%%%%%%%%%%%%%%%%%%%%%%%%%%%%%%%%%%%%%%%%%%%%%%%%%%%%%%%%%%%%%%%%%%
\section{Authors}

\noindent
Rice University's HPCToolkit Research Group \\
Email: \Email{hpctoolkit-forum =at= rice.edu} \\
WWW: \URL{http://hpctoolkit.org}.

\LatexManEnd

\end{document}

%% Local Variables:
%% eval: (add-hook 'write-file-hooks 'time-stamp)
%% time-stamp-start: "setDate{ "
%% time-stamp-format: "%:y/%02m/%02d"
%% time-stamp-end: "}\n"
%% time-stamp-line-limit: 50
%% End:
