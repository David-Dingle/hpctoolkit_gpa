% $Id$

%%%%%%%%%%%%%%%%%%%%%%%%%%%%%%%%%%%%%%%%%%%%%%%%%%%%%%%%%%%%%%%%%%%%%%%%%%%%%
%%%%%%%%%%%%%%%%%%%%%%%%%%%%%%%%%%%%%%%%%%%%%%%%%%%%%%%%%%%%%%%%%%%%%%%%%%%%%

\documentclass[english]{article}
\usepackage[latin1]{inputenc}
\usepackage{babel}
\usepackage{verbatim}

%% do we have the `hyperref package?
\IfFileExists{hyperref.sty}{
   \usepackage[bookmarksopen,bookmarksnumbered]{hyperref}
}{}

%% do we have the `fancyhdr' or `fancyheadings' package?
\IfFileExists{fancyhdr.sty}{
\usepackage[fancyhdr]{latex2man}
}{
\IfFileExists{fancyheadings.sty}{
\usepackage[fancy]{latex2man}
}{
\usepackage[nofancy]{latex2man}
\message{no fancyhdr or fancyheadings package present, discard it}
}}

%% do we have the `rcsinfo' package?
\IfFileExists{rcsinfo.sty}{
\usepackage[nofancy]{rcsinfo}
\rcsInfo $Id$
\setDate{\rcsInfoLongDate}
}{
\setDate{2021/09/15}
\message{package rcsinfo not present, discard it}
}

\setVersionWord{Version:}  %%% that's the default, no need to set it.
\setVersion{=PACKAGE_VERSION=}

%%%%%%%%%%%%%%%%%%%%%%%%%%%%%%%%%%%%%%%%%%%%%%%%%%%%%%%%%%%%%%%%%%%%%%%%%%%%%
%%%%%%%%%%%%%%%%%%%%%%%%%%%%%%%%%%%%%%%%%%%%%%%%%%%%%%%%%%%%%%%%%%%%%%%%%%%%%

\begin{document}

\begin{Name}{1}{hpcstruct}{The HPCToolkit Performance Tools}{The HPCToolkit Performance Tools}{hpcstruct:\\ Recovery of Static Program Structure}

\Prog{hpcstruct} - recovers the static program structure of CPU or GPU binaries, as cited in a recorded
measurements directory, or in a single CPU or GPU binary.
A binary's static structure includes information about source files, procedures,
inlined functions, loops, and source lines.

See \HTMLhref{hpctoolkit.html}{\Cmd{hpctoolkit}{1}} for an overview of \textbf{HPCToolkit}.

\end{Name}

%%%%%%%%%%%%%%%%%%%%%%%%%%%%%%%%%%%%%%%%%%%%%%%%%%%%%%%%%%%%%%%%%%
\section{Synopsis}

\Prog{hpcstruct} \oOpt{options} \Arg{measurement-directory}

\Prog{hpcstruct} \oOpt{options} \Arg{binary}

%%%%%%%%%%%%%%%%%%%%%%%%%%%%%%%%%%%%%%%%%%%%%%%%%%%%%%%%%%%%%%%%%%
\section{Description}

\Prog{hpcstruct} generates a program structure file for each CPU and GPU
binary referenced by a directory containing HPCToolkit performance
measurements.
\Prog{hpcstruct} is usually applied to an application's HPCToolkit \Arg{measurement-directory}, directing
it to analyze all CPU and GPU binaries measured during an execution.
Although not normally needed, one can apply \Prog{hpcstruct} to recover program
structure for an individual CPU or GPU binary.

Program structure is a mapping from addresses of machine instructions
in a binary to source code contexts; this mapping is used to attribute
measured performance metrics back to source code. A strength of
\Prog{hpcstruct} is its ability to attribute metrics to inlined functions
and loops; such mappings are especially useful for understanding the
performance of programs generated using template-based programming
models.
When \Prog{hpcprof} is invoked on a \Arg{measurement-directory} that contains program structure files,
they will be used attribute performance measurements.

\Prog{hpcstruct} is designed to cache its results so that processing multiple measurement directories
can copy previously generated structure files from the cache rather than regenerating the information.
A cache may be specified either on the command line (see the -C option, below) or
by setting the HPCTOOLKIT_HPCSTRUCT_CACHE environment variable.
If a cache is specified, \Prog{hpcstruct} emits a message for each binary processed saying one of "Added to cache",
"Replaced in cache", "Copied from cache", or "Copied from cache +".  The latter indicates that although the
structure file was found in the cache, it came from an identical binary with a different full path,
so it was edited to reflect the actual path.
"Replaced in cache" indicates that a previous version at that path was replaced.
If the \Arg{--nocache} option is set, \Prog{hpcstruct}  will not use the cache, even if the environment
variable is set, and the message will say "(Cache disabled by user)".
If no cache is specified, the message will say "Cache not specified",
and, unless the --nocache option was set, an ADVICE message urging use of the cache will be written.
Users are strongly urged to use a cache.

When an application execution is measured using \Prog{hpcrun}, HPCToolkit records the name of the application binary,
and the names of any shared-libraries and GPU binaries it used. As a GPU-accelerated application is measured,
HPCToolkit also records the contents of any GPU binaries it loads in the application's measurement directory.

When analyzing a \Arg{measurement-directory},  \Prog{hpcstruct} writes its results into a subdirectory of the directory.
It analyzes the application and all the shared libraries and GPU binaries used during the run.
When rerun against a \Arg{measurement-directory},  \Prog{hpcstruct} will reprocess only those GPU binaries that were
previously processed with a different \Arg{--gpucfg} setting.

To accelerate analysis of a measurement directory, which contains
references to an application as well as any shared libraries
and/or GPU binaries it uses, \Prog{hpcstruct} employs multiple threads by
default.  A pool of threads equal to half of the threads in the CPU set for
the process is used.
Binaries larger than a certain threshold (see the \Arg{--psize} option and
its default) are analyzed using more OpenMP threads than those smaller than
the threshold.
Multiple binaries are processed concurrently.
\Prog{hpcstruct} will describe the actual parallelization and concurrency
used when the run starts.

When analyzing a single CPU or GPU binary \Arg{b}, \Prog{hpcstruct} writes its results to the file
'basename(\Arg{b}).hpcstruct' in the current directory.
When rerun against a single binary, it will reprocess that binary and rewrite
its structure file.

\Prog{hpcstruct} is designed for analysis of optimized binaries created from
C, C++, Fortran, CUDA, HIP, and DPC++ source code. Because \Prog{hpcstruct}'s
algorithms exploit the line map and debug information recorded in an
application binary during compilation, for best results, we recommend that
binaries be compiled with standard debug information or at a minimum,
line map information. Typically, this is accomplished by passing a '-g'
option to each compiler along with any optimization flags. See the
HPCToolkit manual for more information.

%%%%%%%%%%%%%%%%%%%%%%%%%%%%%%%%%%%%%%%%%%%%%%%%%%%%%%%%%%%%%%%%%%
\section{Arguments}

\begin{Description}
\item[\Arg{measurement directory}]
A measurement directory of an application, either GPU-accelerated or not.
Applying \Prog{hpcstruct} to a measurement directory analyzes the application as well as
all shared libraries and GPU binaries measured during the data-collection run.

\item[\Arg{binary}] File containing an executable, a dynamically-linked shared library, or a GPU binary
recorded by HPCToolkit as a program executes.
Note that \Prog{hpcstruct} does not recover program structure for libraries that \Arg{binary} depends on.
To recover that structure, run \Prog{hpcstruct} on each dynamically-linked shared library
or relink your program with static versions of the libraries.  Invoking \Prog{hpcstruct} on a binary
is normally not used.

\end{Description}

Default values for an option's optional arguments are shown in \{\}.

\subsection{Options: Informational}

\begin{Description}

\item[\Opt{-V}, \Opt{--version}]
Print version information.

\item[\Opt{-h}, \Opt{--help}]
Print help message.

\item[\OptArg{-v}{num}, \OptArg{--verbose}{num}]
Generate progress messages to stderr, at verbosity level \Arg{num}. {1}

\end{Description}

\subsection{Options: Using the Cache}
\item[\OptArg{-c}{dir}, \OptArg{--cache}{dir}]
Specify the directory to be used for the structure cache.

\item[\Opt{--nocache}]
Specify the the structure cache is not to be used, even if one is specified by the
HPCTOOLKIT_HPCSTRUCT_CACHE environment variable.

\subsection{Options: Override parallel defaults}
\item[\OptArg{-j}{num}, \OptArg{--jobs}{num}]
Specify the number of threads to be used. \Arg{num}
OpenMP threads will be used to analyze any large
binaries. A pool of \Arg{num} threads will be used to
analyze small binaries.

\item[\OptArg{--psize}{n}]
hpcstruct will consider any binary of at least
\Arg{psize} bytes as large. hpcstruct will use more
OpenMP threads to analyze large binaries than
it uses to analyze small binaries.  {100000000}


\item[\OptArg{-s}{v}, \OptArg{--stack}{v}]
Set the stack size for OpenMP worker threads to <v>.
<v> is a positive number optionally followed by
a suffix: B (bytes), K (kilobytes), M (megabytes),
or G (gigabytes). Without a suffix, <v> will be
interpreted as kilobytes. One can also control the
stack size by setting the OMP_STACKSIZE environment
variable. A '-s <v>' option takes precedence,
followed by OMP_STACKSIZE. {32M}

\subsection{Options: override structure recovery defaults}

\begin{Description}

\item[\OptArg{--cpu}{"yes"/"no"}]
Analyze CPU binaries references in a measurements directory. {"yes"}

\item[\OptArg{--gpu}{"yes"/"no"}]
Analyze GPU binaries references in a measurements directory. {"yes"}

\item[\OptArg{--gpucfg}{"yes"/"no"}]
Compute loop nesting structure for GPU machine code. {"no"}

% \item[\OptArg{-I}{path}, \OptArg{--include}{path}]
% Use \Arg{path} when resolving source file names.
% This option is useful when a compiler records the same filename in different ways within the symbolic information.
% (Yes, this does happen.)
% For a recursive search, append a '+' after the last slash, e.g., \texttt{/mypath/+}.
% This option may appear multiple times.

% \item[\OptArg{-R}{'old-path=new-path'}, \OptArg{--replace-path}{'old-path=new-path'}]
% Replace instances of \Arg{old-path} with \Arg{new-path} in all paths with \Arg{old-path} is a prefix
% (e.g., a profile's load map and source code).
% Use \verb+'\'+ to escape instances of '=' within specified paths.
% This option may appear multiple times.
%
% Use this when a profile or executable references files that have been relocated,
% such as might occur with a file system change.

\end{Description}

\subsection{Options to control output:}
\begin{Description}

\item[\OptArg{-o}{filename}, \OptArg{--output}{filename}]
Write the output to to \Arg{filename}.  This option is only applicable when invoking
\Prog{hpcstruct} on a single binary.

\end{Description}

\subsection{Options for Developers:}

\begin{Description}
% \item[\OptArg{--debug}{<n>}]
% Use debug level <n>.  {1}

\item[\Opt{--pretty-print}]
Add indenting for more readable XML output.

\item[\OptArg{--jobs-struct}{num}]
Use \Arg{num} threads for the program structure analysis phase of \Prog{hpcstruct}.

\item[\OptArg{--jobs-parse}{num}]
Use \Arg{num} threads for the parse phase of \Prog{hpcstruct}.

\item[\OptArg{--jobs-symtab}{num}]
Use \Arg{num} threads for the symbol table analysis phase of \Prog{hpcstruct}.

\item[\Opt{--show-gaps}]
Developer option to
write a text file describing all the "gaps" found by \Prog{hpcstruct},
i.e. address regions not identified as belonging to a code or data segment
by the ParseAPI parser used to analyze application executables.
The file is named \emph{outfile}\File{.gaps}, which by default is
\emph{appname}\File{.hpcstruct.gaps}.

\item[\Opt{--time}]
Display the time and space usage per phase in \Prog{hpcstruct}.

\end{Description}

\begin{Description}

\subsection{Options for internal use only}

\item[\OptArg{-M}{dirname}]
Indicates that hpcstruct was invoked by a script used to process measurement directory
\Arg{measurement-dir}. This information is used to control messages.

\end{Description}


%%%%%%%%%%%%%%%%%%%%%%%%%%%%%%%%%%%%%%%%%%%%%%%%%%%%%%%%%%%%%%%%%%
\section{Examples}

\begin{enumerate}
\item
Assume we have used HPCToolkit to collect performance measurements for the (optimized) CPU binary
\File{sweep3d} and that performance measurement data for the application is in the measurement
directory \File{hpctoolkit-sweep3d-measurements}.
Assume that \File{sweep3d} was compiled with debugging information using the -g compiler flag in addition to any
optimization flags.
To recover program structure in \File{sweep3d} and any shared libraries used during the run
for use with \HTMLhref{hpcprof.html}{\Cmd{hpcprof}{1}}, execute:
\medskip
\begin{verbatim}
    hpcstruct hpctoolkit-sweep3d-measurements
\end{verbatim}
\medskip
The output is placed in a subdirectory of the measurements directory.
\medskip
These program structure files are used to interpret performance measurements in \File{hpctoolkit-sweep3d-measurements}.
\medskip
\begin{verbatim}
    hpcprof hpctoolkit-sweep3d-measurements
\end{verbatim}

\item
Assume we have used HPCToolkit to collect performance measurements for the (optimized) GPU-accelerated
CPU binary \File{laghos}, which offloaded computation onto one or more Nvidia GPUs.
Assume that performance measurement data for the application is in the measurement
directory \File{hpctoolkit-laghos-measurements}.
\medskip
Assume that the CPU code for \File{laghos} was compiled with debugging information using the -g compiler flag in addition to any
optimization flags and that the GPU code the application contains was compiled with line map information (-lineinfo).
\medskip
To recover program structure information for the laghos CPU binary, and any shared libraries it used
during the run, as well as any GPU binaries it used, execute:
\medskip
\begin{verbatim}
    hpcstruct hpctoolkit-laghos-measurements
\end{verbatim}
\medskip
The measurement directory will be augmented with program structure information recovered for the
laghos binary, any shared libraries it used, and any GPU binaries it used.  All will be
stored in subdirectories of the measurements directory.
\medskip
\begin{verbatim}
    hpcprof hpctoolkit-laghos-measurements
\end{verbatim}
\end{enumerate}

%%%%%%%%%%%%%%%%%%%%%%%%%%%%%%%%%%%%%%%%%%%%%%%%%%%%%%%%%%%%%%%%%%
\section{Notes}

\begin{enumerate}

\item For best results, an application binary should be compiled with debugging information.
To generate debugging information while also enabling optimizations,
use the appropriate variant of \verb+-g+ for the following compilers:
\begin{itemize}
\item GNU compilers: \verb+-g+
\item Intel compilers: \verb+-g -debug inline_debug_info+
\item IBM compilers: \verb+-g -fstandalone-debug -qfulldebug -qfullpath+
\item PGI compilers: \verb+-gopt+
\item Nvidia's nvcc: \\
~~~~\verb+-lineinfo+ provides line mappings for optimized or unoptimized code\\
~~~~\verb+-G+ provides line mappings and inline information for unoptimized code
\end{itemize}

\item While \Prog{hpcstruct} attempts to guard against inaccurate debugging information,
some compilers (notably PGI's) often generate invalid and inconsistent debugging information.
Garbage in; garbage out.

\item C++ mangling is compiler specific. On non-GNU platforms, \Prog{hpcstruct}
tries both platform's and GNU's demangler.

\end{enumerate}

%%%%%%%%%%%%%%%%%%%%%%%%%%%%%%%%%%%%%%%%%%%%%%%%%%%%%%%%%%%%%%%%%%
%% \section{Bugs}
%%
%% \begin{enumerate}

%% \item xxxxxx

%% \end{enumerate}

%%%%%%%%%%%%%%%%%%%%%%%%%%%%%%%%%%%%%%%%%%%%%%%%%%%%%%%%%%%%%%%%%%
\section{See Also}

\HTMLhref{hpctoolkit.html}{\Cmd{hpctoolkit}{1}}.

%%%%%%%%%%%%%%%%%%%%%%%%%%%%%%%%%%%%%%%%%%%%%%%%%%%%%%%%%%%%%%%%%%
\section{Version}

Version: \Version

%%%%%%%%%%%%%%%%%%%%%%%%%%%%%%%%%%%%%%%%%%%%%%%%%%%%%%%%%%%%%%%%%%
\section{License and Copyright}

\begin{description}
\item[Copyright] \copyright\ 2002-2022, Rice University.
\item[License] See \File{LICENSE}.
\end{description}

%%%%%%%%%%%%%%%%%%%%%%%%%%%%%%%%%%%%%%%%%%%%%%%%%%%%%%%%%%%%%%%%%%
\section{Authors}

\noindent
Rice University's HPCToolkit Research Group \\
Email: \Email{hpctoolkit-forum =at= rice.edu} \\
WWW: \URL{http://hpctoolkit.org}.

\LatexManEnd

\end{document}

%% Local Variables:
%% eval: (add-hook 'write-file-hooks 'time-stamp)
%% time-stamp-start: "setDate{ "
%% time-stamp-format: "%:y/%02m/%02d"
%% time-stamp-end: "}\n"
%% time-stamp-line-limit: 50
%% End:
